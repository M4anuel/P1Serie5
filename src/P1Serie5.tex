\documentclass{article}

\author{Manuel Flückiger 22-112-502 & Abdihakin Sahal Omar 20-947-107}
\title{P1 Serie 04 Theorie Aufgaben}
\date{\today}

\begin{document}
    \maketitle
    \section{Es sein num1=x, num2=y
    \newline Was ist die Ausgabe des folgenden Code Fragments für diese fünf Fälle?}
    \begin{enumerate}
        \item [a)]1 2 3 4 8
        \item [b)]1 2 3 4 8
        \item [c)]3 4 8
        \item [d)]3 5 6 8
        \item [e)]3 7 8
    \end{enumerate}
    \section{Methode um zu entscheiden, ob ein Jahr gregorianisch ist:}
    \begin{verbatim}
    public static boolean gregorianYear(int year){
        if(year<1582){
            System.out.println("Der Kalender war noch nicht geboren");
            return false;
        }
        return (year%4==0&&(year%100!=0||year%400==0));
    }
    \end{verbatim}
    \section{Schreiben Sie eine Methode isIsosceles, die drei ganze Zahlen als
    Parameter entgegennimmt (die L ̈angen der drei Seiten eines Dreiecks).
    Die Methode  gibt true zurück,  falls  das  Dreieck  gleichschenklig
    aber nicht gleichseitig ist (also nur dann, wenn genau zwei Seiten gleich lang sind).}
    \begin{verbatim}
    public static boolean isIsosceles(int x, int y, int z){
        return (!(x == y && y == z ) && (x == y || y == z || z == x ));
    }
    \end{verbatim}
    \section{ Schreiben Sie eine Methode countA, die in einem als Parameter
    übergebenem String name die Anzahl der Zeichen'a' zählt und diese Anzahl zurückgibt}
    \begin{verbatim}
    public static int count(String name, char c){
        return name.length()==0 ? 0 : (name.charAt(0)==c ? 1 : 0) +
                count(name.substring(1),c);
    }
    \end{verbatim}
    in diese Methode kann nach jedem char gesucht werden. Um nach 'a' zu
    suchen muss man count("string", 'a'); eingeben.
    \section{Welche Ausgabe erzeugt folgendes Code-Fragment?}
    0\newline
    1\newline
    2\newline
    3\newline
    4\newline
    \section{Welche Ausgabe erzeugt folgendes Code-Fragment?}
    1 1\newline
    2 2\newline
    2 1\newline
    3 3\newline
    3 2\newline
    3 1\newline
    \section{Schreiben Sie das folgende Code-Fragment mit Hilfe einer while-Schleife um.}
    \begin{verbatim}
        int value = 0;
        int num = 10;
        while(num<=40){
            value+=num;
            num+=10;
        }
        System.out.println(value);
    \end{verbatim}
\end{document}
